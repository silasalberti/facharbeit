\documentclass[a4paper,10pt,ngerman]{scrartcl}
\usepackage{babel}
\usepackage[T1]{fontenc}
\usepackage[utf8x]{inputenc}
\usepackage{csquotes}
\MakeOuterQuote{"}
%\usepackage[a4paper,margin=2.5cm]{geometry}

\usepackage{amsmath}

\title{Die Heisenberg'sche Unschärferelation anhand von diskreten Wahrscheinlichkeitsverteilungen}
\author{Silas Alberti}
\date{Februar 2018}

\begin{document}

\maketitle

\tableofcontents

\section{Einleitung}

Dank der heutzutage weit fortgeschrittenen Displaytechnologie in Form von UHD-Auflösungen und 4K-Fernsehern vergisst man schon fast, dass auch die geradezu lebensecht darstellenden Monitore, das Bild dennoch aus einer Vielzahl an einzelnen Pixeln zusammensetzen. Dem entgegen stellt sich die sogenannte Pixel-Art, die gezielt niedrigere Auflösungen als Stilmittel verwendet. Dabei setzt der Pixel-Artist das Bild meist akribisch Pixel für Pixel zusammen ohne von den Vorteilen höherer Bearbeitungsebenen zu profitieren, die moderne Software bietet.

% Bild: Pixel-Art, z.B. Künstler-Grupp eBoy%

Sehr interessant sind die Einschränkungen, die diese Form der Bilderstellung dem Künstler schafft. So sind elementare Elemente wie gerade Linien nur parallel zu den Linien des Rasters und je nach Auffassung noch diagonal möglich. Erst recht problematisch wird es für komplexere Formen wie z.B. Kreise, die sich zwar mehr und mehr approximieren, aber tatsächlich nie perfekt umsetzen lassen. Die Pixel-Art-Szene hat aus dem Streben nach der bestmöglichen Approximation von Kreisen, Linien und ähnlichem bereits eine eigene Wissenschaft gemacht.

% Bild: Kreis (oder Linien), ins Abbildungsverzeichnis? %

In der Realität außerhalb er Computer-Bildschirme hingegen -- so könnte man sich im Angesicht dieser Unvollkommenheit zureden -- existieren solche Probleme nicht; man könne ja unendlich weit "reinzoomen" in unser Universum. Derartige Überlegungen finden ihren Höhepunkt z.B. in der Analysis, genauer in der Infinitesimalrechnung, die mathematische Sachverhalte bis in unendliche Genauigkeit darstellen können. 

Der elementare Gegensatz sind hier die mathematische Begriffe \textit{diskret} und \textit{stetig} bzw. \text{kontinuierlich}. Ein Computer-Bildschirm der aus einer endlichen Anzahl an Pixeln besteht kann nur diskretes fassen, während unser Universum scheinbar auch kontinuierliche Größen fassen kann. Mathematisch präziser formuliert, betrachtet die 

Als Albert Einstein gegen Anfang des 20. Jh. bisher unerklärbare Phänomene bei der Beobachtung des photoelektrischen Effekts auf ein Lichtquant zurückführte, heute auch Photon genannt.

\section{Hauptteil}

\section{Perspektive}

\section{Anhang}

\subsection{Straightforwardness von Beweisen}

Im Laufe der Arbeit wird mehrmals das englische Wort "straightforward" verwendet, um die Auslassung von Beweisen zu begründen. Grob übersetzt bedeutet das Wort "trivial" oder auch "geradlinig durchführbar". Da ich das englische Wort als so treffend, jedwede deutsche Übersetzung jedoch für unpassend empfand, habe ich es wörtlich verwendet. Hier werde ich kurz erläutern, warum und in genau welchen Fällen ich mich dafür entschieden habe Beweise auszulassen.

Als \textit{straightforward} bezeichne ich Beweise deren Herausforderung -- wenn sie nicht sowieso trivial sind -- nicht wie häufig in dem Finden einer kreativen bis zu genialen Idee liegt, sondern stattdessen in der Komplexität der Notation. Mathematische Notation ist stets nur der Annäherungsversuch an natürliche bzw. zugrundeliegende Wahrheiten oder Gesetzmäßigkeiten, die schwierig zu fassen sind. Manchmal wirken daher Zusammenhänge durch die Verschleierung der Notation als nicht-trivial, obwohl sie aus anderen Blickwinkeln betrachtet naheliegend bis zu offensichtlich sind. Beispielhaft lässt sich die Kettenregel in der Analysis nennen, die in Differentialquotienten (Leibniz-Notation) dargestellt naheliegend scheint,
\[ \frac{\mathrm{d}y}{dx}\frac{dx}{dz} = \frac{dy}{dz},\]
in der Lagrange-Notation dagegen nicht,
\[ (f\circ g)'(x) = f'(g(x))g'(x). \]
Des Weiteren ist der formale Beweis der Kettenregel mittels der h-Methode wesentlich komplizierter als das intuitive Verständnis des zugrundeliegenden Zusammenhangs, ohne jedoch weitere wesentliche Erkenntnis für letzteres zu bieten. Ich möchte nicht behaupten, dass ein genaues Studium diese Beweises überhaupt keine Erkenntnis bieten kann, sondern nur, dass diese Erkenntnis sehr verschlüsselt ist und durch ein anderes Medium wahrscheinlich besser vermittelt werden kann.

Keinesfalls möchte ich die Wichtigkeit von Beweisen per se bestreiten und, um mit gutem Gewissen die Wahrheit der in der Arbeit behandelten Sachverhalte behaupten zu können, habe ich jeden der nicht angeführten Beweise selbst ausgearbeitet. Das ist wichtig, denn oft fallen einem im Beweisprozess Ausnahmen und Einschränkungen der zu zeigenden Aussage auf, die man vorher unter Umständen übersehen hatte. Für wahrhaftige Vollständigkeit müsste wohl jeder der ausgelassenen Beweise elaboriert werden, damit unter voller mathematischer Strenge von der Wahrheit der zu beweisenden Aussagen ausgegangen werden kann. 

In dieser Arbeit ist dies jedoch nicht das Ziel; stattdessen soll sie dem Leser einen direkteren Mehrwert bieten. Technische Termumformungen u.ä. ohne signifikante Erkenntnis beim erstmaligen Betrachten stören lediglich den Lesefluss sowie die Dichte von relevanter Information in der Arbeit -- mit hoher Wahrscheinlichkeit würden sie in den meisten Fällen sowieso übersprungen werden. Nicht zuletzt auch noch wegen der Längenbeschränkung auf 12 Seiten, habe ich mich daher dazu entschlossen, diejenigen Beweise, die -- in meinen Augen -- straightforward sind, auszulassen.

\subsection{Abbildungen}

\subsection{Literatur}

\end{document}
